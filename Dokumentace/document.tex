\documentclass[12pt,a4paper]{article}
\usepackage[utf8]{inputenc}
\usepackage[czech]{babel}
\usepackage[T1]{fontenc}
\usepackage{amsmath}
\usepackage{hyperref}
\usepackage{amsfonts}
\usepackage{amssymb}
\usepackage{graphicx}
\usepackage{enumitem}


\begin{document}
	\begin{titlepage}
		\begin{center}
			\Huge
			\textbf{HLEDÁNÍ MAXIMÁLNÍHO TOKU V GRAFU SILNIČNÍ SÍTĚ} \\
			
			\bigskip  \bigskip
			
			
			\huge
			Semestrální práce
			\
			
			\bigskip
			\Large{Kryštof Šaml} \\
			\bigskip
			\today
			
			\begin{figure}[h]
				
				\centering
				\includegraphics[width=120mm]{FAV_logo.png}
				\label{overflow}
				
				
			\end{figure}
		\end{center}
	\end{titlepage}
	\newpage
	\tableofcontents
	\newpage
	
	\section{Zadání}
	Naprogramujte v jazyce ANSI C přenositelnou \textbf{konzolovou aplikaci}, jejímž účelem bude nalezení maximálního toku v síti $\vec{G}$ s jedním zdrojem \textit{z} a jedním stokem \textit{s}. Vaše vstupní i výstupní soubory typu \textbf{csv} je možné vizualizovat pomocí aplikace \textit{QGIS}, která je dostupná na adrese \\[\baselineskip] \url{https://qgis.org/en/site/forusers/download.html}.\\[\baselineskip] Program se bude spouštět příkazem \textbf{flow.exe} s následujícími přepínači:

	
	\setlist[description]{labelindent=12pt,style=multiline,leftmargin=4cm}
	\begin{description}
	\item[-v \textit{\textless uzly.csv\textgreater}] Parametr specifikuje vstupní soubor typu \textbf{csv}, který obsahuje informace o uzlech sítě. Při zadání nevalidní tabulky program vypíše chybovou hlášku {\textbf{„Invalid vertex file.\textbackslash n"}} a skončí s návratovou hodnotou 1. Zaručte,
	aby načtení uzlů grafu bylo vždy první operací.
	\item[-e \textit{\textless hrany.csv\textgreater}] Parametr určuje vstupní soubor typu {\textbf{csv}}, jehož řádky popisují hrany sítě. V případě zadání nevalidní tabulky program vypíše chybovou hlášku {\textbf{„Invalid edge file.\textbackslash n"}} a skončí s návratovou hodnotou 2.
	\item[-s \textit{\textless source\_id\textgreater}] Parametr jednoznačně určuje zdroj v síti díky primárnímu klíči id tabulky \textit{\textless uzly.csv\textgreater} . Pokud zadaný uzel v tabulce \textit{\textless uzly.csv\textgreater} neexistuje, program vypíše chybovou hlášku \textbf{„Invalid source vertex.\textbackslash n"} a skončí s návratovou hodnotou 3.
	\item[-t \textit{\textless target\_id\textgreater}] Parametr na základě primárního klíče \textbf{id} v tabulce \textit{\textless uzly.csv\textgreater} jednoznačně určuje stok v síti. Pokud uvedený uzel v tabulkce \textit{\textless uzly.csv\textgreater} neexistuje, nebo je shodný se zdrojem \textit{\textless source\_id\textgreater} , program vypíše chybovou hlášku \textbf{„Invalid sink vertex.\textbackslash n"} a skončí s návratovou hodnotou 4.
	\newpage
	\item[-out \textit{\textless out.csv\textgreater}] V případě úspěšného nalezení toku nenulové velikosti jsou do csv souboru daného tímto \textit{nepovinným} parametrem uloženy hrany minimálního řezu, jež odděluje zadaný zdroj a stok. Výstupní tabulka má stejnou strukturu jako vstupní (\textit{\textless hrany.csv\textgreater}). Hrany jsou v této tabulce seřazeny dle jejich identifikátoru \textbf{id} vzestupně. Pokud je zadané umístění neplatné, program vypíše chybovou hlášku \textbf{„Invalid output file.\textbackslash n"} a skončí s návratovou hodnotou 5. Pokud soubor již existuje, program jej přepíše.
	
	\item[-a] Jedná se o \textit{nepovinný} přepínač, při jehož uvedení bude program pracovat s hranami, které mají příznak \textbf{isvalid} nastavený na hodnotu \textbf{False}. Při spuštění bez parametru \textbf{-a} tedy program uvažuje pouze hrany s příznakem \textbf{isvalid} s hodnotou \textbf{True}.
	
	\end{description}
	
	\begin{itemize}
	\item[]
	\end{itemize}
	
	Není-li některý z povinných parametrů uveden, aplikace končí příslušnou chybovou hláškou a návratovou hodnotou. Po úspěšném dokončení algoritmu hledání maximálního toku program vypíše hlášku \textbf{„Max network flow is\textbar x\textbar =\textless s\textgreater .\textbackslash n"}, kde \textless s\textgreater je velikost nalezeného toku, tj. propustnost minimálního řezu. V případě, že v síti neexistuje tok nenulové velikosti, program skončí s návratovou hodnotou 6. Platí-li opačné tvrzení a zároveň uživatel použil parametr -out, pak je vytvořen výstupní soubor \textit{\textless out.csv \textgreater} , do kterého jsou zapsány hrany odpovídajícího minimálního řezu. Program následně skončí s návratovou hodnotou \textbf{EXIT\_SUCCESS}. \\[\baselineskip]
	Váš program může být během testování spuštěn například takto:
	\\[\baselineskip]
	\centerline{\textbf{flow.exe -e edgs.csv -v "vertcs file.csv" -a -s 43 -t 81}}
	\newpage
	\section{Analýza úlohy}
	Ze zadání je jasné, že jde o grafový problém. Konkrétně o nalezení maximálního toku v síti s jedním zdrojem a s jedním stokem.
	\bigskip
	\subsection{Specifikace vstupních dat}
	Uzly a hrany sítě jsou uloženy v samostatných vstupních souborech typu \textbf{csv}. Pro oddělení jednot livých sloupců se využívá znak čárky (,). Tabulky uzlů a hran mají následující společné atributy: \bigskip
	\setlist[description]{labelindent=12pt,style=multiline,leftmargin=2.5cm}
	\begin{description}
	\item[id] Účelem tohoto sloupce je jednoznačná identifikace uzlu nebo hrany sítě. Je možné, že kvůli chybám při tvorbě datasetu není tento identifikátor unikátní. Při načítání dat tedy berte v potaz pouze první výskyt uzlu nebo hrany s daným identifikátorem a ostatní ignorujte.
	\item[WKT] Obsahem tohoto sloupce je geografický popis objektu. Bod je popsán zeměpisnou šířkou a délkou. Hrany (pozemní komunikace) jsou dány jako lomené čáry procházející sérií bodů. Je třeba věnovat pozornost tomu, že pro oddělení jednotlivých bodů lomené čáry se opět využívá znak čárky (,). Jedná se však stále o jeden atribut. Právě díky tomuto sloupci je možné graf vykreslit v mapových podkladech pomocí softwaru QGIS.
	\end{description}
	
	\begin{itemize}
	\item[]
	\end{itemize}
	
	 Atributy konkrétních tabulek jsou uvedeny v textu níže. Hodnoty všech vstupních sloupců musejí být přítomny i ve výstupních souborech, a to v nezměněné podobě.
	 \\[\baselineskip]
	 \subsubsection{Tabulka uzlů sítě}
	 Tabulka uzlů obsahuje uspořádanou dvojici sloupců id a WKT. Pokud na vstupu bude zadána tabulka s jinými atributy, aplikace skončí s výše popsaným chybovým hlášením. Rovněž při tvorbě nového souboru je třeba tuto strukturu zachovat.
	 \\[\baselineskip]
	 \subsubsection{Tabulka hran sítě}
	 Tabulka hran obsahuje uspořádanou množinu sloupců id, source, target, capacity, isvalid a WKT. Pokud na vstupu bude zadána tabulka s jinými atributy, aplikace skončí s popsaným chybovým hlášením. Rovněž při tvorbě nového souboru je třeba tuto strukturu dodržet. Orientovaná hrana je definována mezi uzly, jejichž identifikátory jsou uvedeny ve sloupcích source a target. Teoretická propustnost hrany (pozemní komunikace) je uvedena ve sloupci capacity.

	 
	 \subsection{Reprezentace grafu}
	 Graf se dá v počítači reprezentovat dvěma způsoby. Prvním způsobem je matice sousednosti. Tento způsob je vhodný pro grafy s velkým množstvím hran. Výhodou je $O(1)$ časová náročnost při vyhledávání hran v grafu. Nevýhodou je paměťová náročnost $O(v^2)$, kde v je počet uzlů. Pro nízké množství hran je zbytečně paměťově náročný. Z toho důvodu jsem zvolil druhý způsob, a to pomocí seznamů sousednosti. Výhodou je nižší paměťová náročnost. Nevýhodou může být pomalé vyhledávání hran v grafu, to ale v tomto případě nevadí, protože není třeba hrany vyhledávat.
	 \subsection{Algoritmy pro hledání maximálního toku v síti}
	 Algoritmy pro hledání maximálního toku v síti fungují v několika krocích. Všechny jsou si podobné, tak popíši hlavně jejich rozdíly. Je dána síť $G=(V,E)$ se zdrojem \textit{s} a stokem \textit{t}.
	 \begin{enumerate}
	 \item f$(u,v)$ přiřaď 0 všem hranám $(u,v)$.
	 \item Dokud existuje cesta \textit{p} z \textit{s} do \textit{t} v $G$, taková, že platí $c(u,v)\textgreater$  0 pro všechny hrany $(u,v)\in p$
	 \begin{enumerate}
	 \item najdi $c_f(p)=\min\{c_f(u,v):(u,v)\in p\}$
	 \item Pro každou hranu $(u,v)\in p$:
	 \begin{enumerate}
	 \item $f(u,v)$ přiřaď $f(u,v)+c_{f}(p)$ 
	 \item $f(v,u)$ přiřaď $f(v,u)-c_{f}(p)$ (jde o zpětnou hranu)
	 \end{enumerate}
	 \end{enumerate}
	 \end{enumerate}
	 \subsubsection{Ford-Fulkerson}
	 Algoritmus má časovou náročnost $O(f\cdot E)$, kde f je maximální tok a E počet hran. Pro vyhledávání cesty využívá prohledávání do hloubky, pokud se v grafu vyskytuje hrana s nízkou kapacitou, tak může průběh algoritmu velmi zpomalit. Obecně je závislost algoritmu na velikosti maximálního toku velmi negativní. Algoritmus je pak pomalý.
	 \subsubsection{Edmonds-Karp}
	 Algoritmus má časovou náročnost $O(E^2\cdot V)$, kde E je počet hran a V je počet uzlů. Pro vyhledávání cesty využívá prohledávání do šířky. Jeho časová náročnost už není závislá na velikosti maximálního toku. Je obecně rychlejší než Ford-Fulkerson.
	 \subsubsection{Škálování kapacity}
	 (v originále Capacity Scaling) 
	 Algoritmus má časovou náročnost $O(E^2\cdot log(u))$, kde E je počet hran a U je hodnota nejvyšší kapacity ze všech hran v celém grafu. Jde o vylepšenou verzi Ford-Fulkersonova algoritmu, ve které je heuristika, která se snaží o využívání hran s vyšší kapacitou oproti hranám s nižší kapacitou. Definujme si $\delta$
	 , která bude rovna nejvyšší mocnině dvou, která je menší nebo rovna hodnotě U. Heuristika říká, že bychom něměli využívat hrany s kapacitou menší než $\delta$. Pokud už žádné takové nejsou, tak $\delta = \delta \div 2$, algoritmus pokračuje dokud $\delta \textgreater  0$ Algoritmus je opět rychlejší než oba dva předchozí.
	 \newpage
	 \subsubsection{Dinicův (též Dinitzův) algoritmus} 
	 Algoritmus má časovou náročnost $O(V^2\cdot E)$, kde E je počet hran a V je počet uzlů. Algoritmus využívá tzv. úrovňový graf (viz obrázek). Ten se vytvoří použitím prohledávání do šířky, každý uzel grafu je přiřazen do úrovně podle toho kolik hran se musí projít aby bylo možné se ze zdroje dostat do daného uzlu. Když je úrovňový graf připraven, tak se pomocí prohledávání do hloubky hledá cesta ze zdroje do stoku s tím, že se využívají pouze hrany, které vedou do vyšší úrovně grafu. Když už žádné takové cesty nejsou (dosáhne se tzv. blokujícího toku), tak se znovu vytvoří úrovňový graf a pak se znovu hledají cesty ze zdroje do stoku s použitím prohledávání do hloubky. To se děje dokud je možné se dostat s pomocí prohledávání do šířky ze zdroje do stoku. Algoritmus se ještě dá optimalizovat tím, že když se dostane do "slepé uličky", tak hrany, které do ní vedly už nepoužije (dokud se nevytvoří nový úrovňový graf). Tento algoritmus je ze všech 4 zmíněných bezkonkurenčně nejrychlejší. Proto jsem si ho zvolil.
	 \bigskip
	 
	 \centerline{\includegraphics[width=120mm]{level_Graph.png}}
	 \label{Obrázek 1}
	 \section{Popis implementace}
	 \subsection{Datové struktury}
	 \subsubsection{Bod}
	 Bod má celočíselnou proměnnou \textbf{id}, která nese unikátní hodnotu identifikátoru, celočíselnou proměnnou \textbf{next}, která reprezentuje index první hrany vedoucí z tohoto bodu, která nevede do "slepé uličky", celočíselnou proměnnou \textbf{level}, která říká, kolik hran musíme ze zdroje projít, abychom se dostali do tohoto bodu a řetězec \textbf{wkt}, který popisuje geografickou polohu bodu.
	 \subsubsection{Hrana}
	 Hrana obsahuje celočíselné proměnné \textbf{id, source, target, flow a capacity}, dále obsahuje znak \textbf{isValid}, ukazatel \textbf{reverseEdge} a řetězec textbf{wkt}. \textbf{Id} nese unikátní hodnotu identifikátoru, \textbf{source} je id uzlu, ze kterého hrana vede, \textbf{target} je id uzlu do kterého hrana vede, \textbf{flow} je aktuálně využitá kapacita hrany, \textbf{capacity} je maximální možnou kapacitou hrany, \textbf{reverseEdge} je ukazatel na hranu vedoucí v opačném směru, \textbf{isValid} nabývá hodnot \textbf{‘Y‘} nebo \textbf{‘N‘} podle toho jestli je validní, \textbf{wkt} popisuje geografickou polohu hrany. K hraně patří funkce na její vytvoření, která  z řetězce předaného jako parametr vytvoří novou hranu a funkce na uvolnění paměti.
	 \subsubsection{Fronta}
	 Fronta obsahuje celočíselné proměnné \textbf{capacity, first a count}. Dále obsahuje pole celých čísel \textbf{array}. \textbf{Capacity} je velikost pole array. \textbf{First} je index prvního prvku ve frontě. \textbf{Count} je počet aktuálně obsazených prvků, s pomocí first, count a capacity se dá zjistit index na který se přidá další prvek. Pole \textbf{array} obsahuje všechny prvky uložené ve frontě. K frontě patří funkce na vytvoření fronty, přidání prvku do fronty, odebrání prvního prvku z fronty, funkce na zvětšení kapacity fronty, testovací funkce zda je fronta prázdná, zda je fronta plná a funkce na uvolnění paměti.
	 \subsubsection{Dynamické pole}
	 Dynamické pole obsahuje celočíselné proměnné \textbf{listSize, itemSize a filledItems}, také obsahuje pole ukazatelů \textbf{data}. \textbf{ListSize} je velikost pole data, \textbf{itemSize} je velikost položky, \textbf{filledItems} je počet obsazených prvků v poli data a index prvního volného ukazatele v poli data (pouze pokud se nevyužívají funkce tabulky). Dyn. Patří k němu funkce na vytvoření, uvolnění paměti, přidání prvku, vybrání prvku z indexu a zvětšení pole ukazatelů.
	 \subsubsection{Tabulka}
	 Tabulka obsahuje celočíselné proměnné \textbf{size, itemSize, filledItems} a pole dynamických polí \textbf{array}. \textbf{Size} je velikost pole \textbf{array}, \textbf{itemSize} je velikost položky uložené v tabulce a \textbf{filledItems} je počet aktuálně obsazených položek v tabulce. Patří k ní funkce na vytvoření tabulky, uvolnění paměti zabrané tabulkou, přidání prvku, zjištění jestli už tabulka daný prvek obsahuje, získání prvku s daným klíčem.
	 \subsubsection{Graf}
	 Graf obsahuje celočíselné proměnné \textbf{maxFlow, source, target}, tabulku \textbf{nodes} ve které jsou uloženy uzly a tabulku \textbf{edges} ve které jsou uloženy hrany grafu. \textbf{Nodes} obsahuje všechny uzly grafu (u každého uzlu jsou i důležité proměnná level a next). \textbf{MaxFlow} je proměnná do které se uloží maximální tok v síti. \textbf{Source} označuje index uzlu zdroje, \textbf{target} označuje index uzlu stoku. V tabulce \textbf{edges} jsou uloženy všechny hrany grafu. Proměnná \textbf{level} jednotlivých uzlů slouží pro uložení minimálního počtu hran, které se musí projít, aby se dalo dostat z uzlu source do daného uzlu, hodnoty jsou v průběhu algoritmu aktualizovány. Proměnná \textbf{next} jednotlivých uzlů slouží pro uložení indexu první neprozkoumané hrany, která nekončí “slepou uličkou“, hodnoty jsou v průběhu algoritmu aktualizovány. Ke grafu patří funkce na vytvoření grafu, na práci s hranami, na vypočtení maximálního toku a na uvolnění paměti využívané grafem.
	 \subsection{Ukládání dat}
	 Pro ukládání dat jsem zvolil tabulky. S jejich použitím je jednoduché se zbavit duplicitních hran, pokud se jako klíč do tabulky použije id.
	 \subsection{Vytvoření grafu}
	 Pro reprezentaci grafu jsem využil seznam sousednosti kvůli jeho menší paměťové náročnosti. Při vytváření grafu se využije celá tabulka s načtenými uzly. Tabulka s hranami se nevyužije, protože hrany jsou na nesprávných indexech (jsou na indexech daných jejich id, mají být na indexech, které odpovídají jejich proměnné source) , pak následuje sekvenční načítání hran. V každém kroku je načtena hrana, vytvoří se k ní hrana vedoucí v opačném směru s nulovou kapacitou, obě dvě hrany jsou uloženy mezi hrany grafu na indexy, které odpovídají hodnotě jejich proměnné \texttt{source}. Dále se inicializují hodnoty \texttt{maxFlow} na 0, \texttt{source} na -1 a \texttt{target} na -1. Hodnoty proměnných \texttt{level} a \texttt{next} jednotlivých uzlů se budou upravovat až v průběhu vyhledávání maximálního toku v síti.
	 \newpage
	 \subsection{Nalezení maximálního toku}
	 Pro nalezení maximálního toku jsem využil Dinicův algoritmus. Nejdřív však musím vysvětlit smysl proměnných level a next u jednotlivých uzlů. V proměnné level je uložena informace kolik hran musím projít, abych se dostal z uzlu source do daného uzlu. proměnná next pomáhá algoritmus optimalizovat tím, že je v ní uložen index první neprozkoumané hrany, která nekončí “slepou uličkou“, je to pomůcka pro DFS, aby zbytečně víckrát neprocházelo cestu, která nevede k cíli. Dinicův algoritmus pracuje s využitím prohledávání do hloubky, ale i s využitím prohledávání do šířky. Na začátku se nastaví source a target na požadované uzly zdroje a stoku. Vytvoří se Fronta, poté se spustí hlavní cyklus, který běží do té doby dokud funkce graphBfs nalézá rozšiřující cestu ze zdroje source do stoku target. V hlavním cyklu se vždy vynulují proměnné next u všech uzlů grafu a pak se hledají rozšiřující cesty ze zdroje do stoku, dokud existují. Jejich hledání zajišťuje funkce graphDfs.
	 \subsubsection{GraphBfs}
	  Při každém volání nejdříve nastaví v pro všechny uzlu hodnotu proměnné level na -1, která říká, že uzel ještě nebyl navštíven. Poté nastaví proměnnou level uzlu source na 0, tedy k dosažení tohoto uzlu z uzlu source je třeba 0 hran. Přidá source do Fronty a pak se spustí cyklus, který běží dokud Fronta není prázdná. V cyklu se nejdříve vyjme první hodnota z Fronty, řekněme, že to bude X. Pak se projdou všechny hrany vedoucí z uzlu X, pokud se mezi nimi najde taková, která má ještě nějakou volnou kapacitu a zároveň uzel do kterého hrana vede ještě nebyl navštíven, řekněme, že to bude Y (hodnota proměnné level uzlu Y musí být -1), potom se Y přidá do Fronty a hodnota proměnné level na se nastaví na (hodnotu level uzlu X + 1). Až se Fronta úplně vyprázdní, tak se vrátí hodnota, která říká jestli byl uzel target navštíven. 
	  \subsubsection{GraphDfs}
	   GraphDfs je rekurzivní metoda, když zjistí, že její parametr node se rovná stoku grafu target, tak vrátí hodnotu flow. V opačném případě prochází hrany vycházející z uzlu node (a to od indexu který udává proměnná next uzlu node). Pokud pro některou z hran platí (řekněme že jde o hranu E), že má ještě nevyužitou kapacitu a zároveň proměnná level uzlu node je o jedna nižší než proměnná level uzlu (koncový uzel hrany E), tak se rekurzivně zavolá s parametry koncového uzlu hrany E, a minima z nevyužité kapacity hrany a aktuální kapacity flow, pokud  hodnota vrácená po rekurzivním volání je větší než 0, tak se o tu hodnotu rozšíří kapacita hrany E a vrátí se vrácená hodnota. Pokud se nenajde hrana E, tak se inkrementuje proměnná next uzlu node (díky tomu se nebude tato hrana procházet vícekrát, protože už víme, že nikam nevede). Pokud už se prošly všechny hrany vedoucí z uzlu node a žádná z nich nevede do uzlu stoku, tak funkce vrací nulu.
	 \section{Uživatelská příručka}
	 \subsection{Složení programu}
	 Program se skládá z šesti hlavičkových souborů a šesti souborů s výkonným kódem. První hlavičkový soubor \textbf{arrayList.h} slouží jako reprezentace dynamického pole s prototypy příslušných funkcí. Výkonný kód k dynamickým polím se nachází v souboru \textbf{arrayList.c}. Druhý hlavičkový soubor \textbf{graph.h} obsahuje prototypy funkcí pro práci s grafem a hranami v grafu. Graf samotný se nachází v jiném hlavičkovém souboru a to v souboru \textbf{structs.h}. Výkonný kód ke grafu se nachází v souboru \textbf{graph.c}.  Třetí hlavičkový soubor \textbf{inputOutput.h} obsahuje prototypy funkcí pro načítání a zapisování dat, výkonný kód se nachází v souboru \textbf{inputOutput.c}. Čtvrtý hlavičkový soubor \textbf{intQueue.h} slouží jako reprezentace fronty s prototypy příslušných funkcí. Výkonný kód k frontě je v souboru \textbf{intQueue.c}. Pátý hlavičkový soubor \textbf{hashTable.h} slouží jako reprezentace tabulky s prototypy příslušných funkcí. Výkonný kód k tabulce je v souboru \textbf{hashTable.c}. Posledním souborem je soubor s výkonným kódem \textbf{main.c}, který obsahuje funkci main.
	 \subsection{Překlad a spuštění programu}
	 Pro překlad potřebujeme mít nainstalovaný překladač jazyka C. Zvolíme například překladač gcc, ten funguje na unixových systémech, ale také v operačních systémech Windows, konkrétně jde o MinGW gcc. Při instalaci na Windows musíme nastavit cestu k překladači. Také musíme nastavit cestu k souboru \textbf{mingw32-make.exe}, který nám pomůže program přeložit. Když máme nastaveno, otevřeme příkazovou řádku ve složce, kde jsou zdrojové soubory. K přeložení použijeme soubory \textbf{Makefile} (pro unix) nebo \textbf{Makefile.win} (pro Windows), zadáme příkaz \texttt{make flow.exe}. Ten už pak můžeme spouštět tak, že napíšeme do příkazové řádky \texttt{flow.exe}, za něj musíme napsat všechny povinné vstupní parametry (ty jsou důkladně rozepsány v části zadání). Pokud byly všechny parametry zadány správně, program vypíše maximální tok v dané síti a pokud jsme si to zvolili v parametrech, tak vytvoří i soubor s hranami minimálního řezu. Pokud parametry nebyly zadány správně, pokud některý z nich chyběl, nebo pokud vstupní soubory nebyly validní, pak se vypíše příslušné chybové hlášení objasňující, v čem byla chyba.
	 \section{Závěr}
	 \subsection{Běh programu}
	 Program jsem testoval na všech přiložených souborech pro testování. Běh programu byl velmi rychlý jak pro sadu s malými testovacími soubory (0 s, respektive něco tak malého, že se to na nulu zaokrouhlilo), tak pro “real world“ sadu s velkými testovacími soubory (zhruba 0,05 s), délka běhu programu mohla být ovlivněna výběrem zdroje a stoku.
	 \subsection{Problémy}
	 Největším problémem samozřejmě bylo podcenění vstupních dat. Kvůli tomu jsem nebyl schopen odevzdat semestrální práci v řádném termínu. To jsem musel řešit pomocí tabulek. Dalším problémem pro mě bylo načítání dat, to, že obsahovala duplicity mi problém nedělalo, spíš jsem nezvládal samotné načtení. To, že jsem tehdy poprvé pořádně viděl jazyk C, tomu moc nepomohlo. Předtím jsem byl zvyklý hlavně na jazyk Java a s ničím jako je C jsem se nepotkal.
	 \subsection{Možná vylepšení}
	 Největším nedostatkem mé práce je jistě metoda pro vytváření hran ze vstupního řetězce. Kód určitě nebude optimální, stejně tak by se dala zlepšit i přehlednost rozdělením některých částí do samostatných funkcí. Dalším nedostatkem je to, že jsem zvyklý na Javu, proto je možné, že některé části kódu budou pro někoho kdo je zvyklý na C vypadat velmi podivně. Jako přirozená cesta mi vždy přišlo dělat to “jako kdybych programoval v Javě“. Dalším nedostatkem může být složitost, která je způsobena úpravami kvůli indexům vstupních dat.
	 \subsection{Shrnutí}
	 Práci jsem se snažil vypracovat jak jen to šlo. Celou práci jsem si výrazně zkomplikoval tím, že jsem se rozhodl využít jiný algoritmus pro hledání maximálního toku. Byl ale o tolik rychlejší, že jsem to chtěl zkusit. Kód ze cvičení jsem skoro nepoužíval, vše jsem si psal znovu sám. Podcenil jsem vstupní data (neprošel jsem posledním testem validátoru s vysokými id vstupních dat). Proto jsem musel svou práci výrazně upravit. Tato úprava podle mě zvětšila složitost kódu. Velmi se mi líbilo, že šlo o reálná data a výstup programu má reálné využití.

\end{document}